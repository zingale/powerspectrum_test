\documentclass[11pt]{article}


\usepackage[margin=1in]{geometry}

\usepackage{mathpazo}

\newcommand{\numpy}{{\sf NumPy}}

\begin{document}

\begin{center}
{\bf \Large Making Sense of Power Spectra}
\end{center}

We are interested in computing the numerical power spectrum of a
function with a known analytic power spectrum, to verify our
algorithm.

Consider a function that is a superposition of sines with different
wavelengths.  The power will be encoded in the amplitude of the sines.
Now imagine this function discretely sampled at $N$ points.  For a domain of
length $L$ the highest possible wavenumber is:
\begin{equation}
k_\mathrm{max} = \frac{1}{\Delta x}
\end{equation}
with $\Delta x = L/N$.  However, for real-valued data, Nyquist sampling
means that $k_\mathrm{max}/2$ is the meaningful maximum.
The lowest wavenumber is:
\begin{equation}
k_\mathrm{min} = \frac{1}{L}
\end{equation}


\section*{\numpy\ details}

The {\tt rfftfreq} function returns the wavenumbers as
\begin{equation}
k = \left \{ 0, \frac{1}{N}, \frac{2}{N}, \ldots, \frac{N-2}{2N}, \frac{1}{2} \right \}
\end{equation}
(for even $n$, and in the mode where we do not specify the sample spacing).
We convert these into a physical quantity (with units of cm$^{-1}$) by
multiplying by $1/\Delta x$, giving the set:
\begin{equation}
k = \left \{ 0, \frac{1}{L}, \frac{2}{L}, \ldots, \frac{1}{2\Delta x} \right \}
\end{equation}

\numpy\ does not have a real-valued FFT function for 3-d data, so we
need to use the {\tt fftn} function.  Because $N$ real numbers results
in $2N$ complex numbers, only 1/2 of the transformed data (in each
dimension) will be unique.  In 3-d this means we consider only an
octant of data.  In practice, this means keeping the positive
frequencies and dropping the negative ones (this is what the {\tt
  rfftfreq} function does for us).


Parseval's theorem in 1-d says that
\begin{equation}
\sum_{n=0}^{N-1} | f(n) |^2 = \frac{1}{N} \sum_{k=0}^{N-1} | \mathcal{F}(k) |^2
\end{equation}
In 3-d, if we have $N^3$ points, then the normalization is $1/N^3$.



\end{document}
