\documentclass[11pt]{article}


\usepackage[margin=1in]{geometry}

\usepackage{mathpazo}

\usepackage{amsmath}

\newcommand{\numpy}{{\sf NumPy}}

\begin{document}

\begin{center}
{\bf \Large Making Sense of Power Spectra}
\end{center}

We are interested in computing the numerical power spectrum of a
function with a known analytic power spectrum, to verify our
algorithm.

Consider a function that is a superposition of sines with different
wavelengths.  The power will be encoded in the amplitude of the sines.
Now imagine this function discretely sampled at $N$ points.  For a domain of
length $L$ the highest possible wavenumber is:
\begin{equation}
k_\mathrm{max} = \frac{1}{\Delta x}
\end{equation}
with $\Delta x = L/N$.  However, for real-valued data, Nyquist sampling
means that $k_\mathrm{max}/2$ is the meaningful maximum.
The lowest wavenumber is:
\begin{equation}
k_\mathrm{min} = \frac{1}{L}
\end{equation}


\section*{\numpy\ details}

The {\tt rfftfreq} function returns the wavenumbers as
\begin{equation}
k = \left \{ 0, \frac{1}{N}, \frac{2}{N}, \ldots, \frac{N-2}{2N}, \frac{1}{2} \right \}
\end{equation}
(for even $n$, and in the mode where we do not specify the sample spacing).
We convert these into a physical quantity (with units of cm$^{-1}$) by
multiplying by $1/\Delta x$, giving the set:
\begin{equation}
k = \left \{ 0, \frac{1}{L}, \frac{2}{L}, \ldots, \frac{1}{2\Delta x} \right \}
\end{equation}

\numpy\ does not have a real-valued FFT function for 3-d data, so we
need to use the {\tt fftn} function.  Because $N$ real numbers results
in $2N$ complex numbers, only 1/2 of the transformed data (in each
dimension) will be unique.  In 3-d this means we consider only an
octant of data.  In practice, this means keeping the positive
frequencies and dropping the negative ones (this is what the {\tt
  rfftfreq} function does for us).


Parseval's theorem in 1-d says that
\begin{equation}
\sum_{n=0}^{N-1} | f(n) |^2 = \frac{1}{N} \sum_{k=0}^{N-1} | \mathcal{F}(k) |^2
\end{equation}
In 3-d, if we have $N^3$ points, then the normalization is $1/N^3$.



\section*{Test function}

To test the algorithm, we define a test function and see the power at
different wavenumbers {\em in physical space}, and then do the transform
and power spectrum to see if we get the expected result.  We will design
our function to have a power spectrum scaling as:
\begin{equation}
P \sim k^{-\eta}
\end{equation}
Our test function
is:
\begin{equation}
\phi(x,y,z) = \sum_{m=1}^{N_\mathrm{modes}} 
              \sum_{n=1}^{N_\mathrm{modes}} 
              \sum_{p=1}^{N_\mathrm{modes}} 
     a_{m,n,p} \sin \left ( \frac{2\pi m x}{L_x} + 
                            \frac{2\pi n y}{L_y} + 
                            \frac{2\pi p z}{L_z} \right )
\end{equation}
where we can identify the wavenumbers in each direction as
\begin{equation}
k_m = \frac{m}{L_x} \, ; \quad 
k_n = \frac{n}{L_y} \, ; \quad 
k_p = \frac{p}{L_z} \,
\end{equation}
This is equivalent to defining a vector wavenumber
\begin{equation}
\vec{k} = k_m \hat{x} + k_n \hat{y} + k_p \hat{z}
\end{equation}
and expressing a single component as
\begin{equation}
\phi_{m,n,p} = a_{m,n,p} \sin(2\pi \vec{k}\cdot \vec{x})
\end{equation}
and then we can set the amplitude of component based on our desired
powerlaw scaling.  Defining $k = |k|$, and $A_0$ as a
reference amplitude of the power, we take
\begin{equation}
a_{m,n,p} = \frac{A_0 k^{-\eta}}{W(k)}
\end{equation}
where $W(k)$ is the weight that accounts for the fact that multiple
combinations of $m, n, p$ can result in the same $k$.  Using Iverson bracket
notation, defined as
\begin{equation}
[P] = \begin{cases}
  1 & \text{if P is true} \\
  0 & \text{otherwise}
      \end{cases}
\end{equation}
we find $W(k)$ as
\begin{equation}
W(k) = \sum_{m=1}^{N_\mathrm{modes}}
       \sum_{n=1}^{N_\mathrm{modes}} 
       \sum_{p=1}^{N_\mathrm{modes}} [ k_m^2 + k_n^2 + k_p^2 = k ]
\end{equation}

\end{document}
